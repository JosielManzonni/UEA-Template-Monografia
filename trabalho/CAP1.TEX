\chapter {Como Escrever Uma monografia}\label{cap:comoEscrever}

\section{Ao Candidato}

O texto abaixo foi livremente adaptado de \cite{COMER2010} para ajudar os alunos a escreverem suas monografias. 

Se você está se preparando para escrever uma monografia em uma área experimental da Engenharia da Computação. 
A menos que você tenha escrito muitos documentos formais antes, você tem uma surpresa: isso é difícil!

Existem dois caminhos possíveis para o sucesso:

\begin{itemize}
	\item Planejamento
	
	Poucas pessoas pegam esse caminho. As poucas que pegam, deixam a universidade tão rápido, 
	que eles malmente são notados. Se você quer fazer a impressão final e ter uma longa carreira como um 
	estudante de graduação, não escolha este caminho.
	
	\item Perseverança
	
	Tudo que você tem que fazer é sobreviver à sua banca julgadora. 
	A boa notícia é que eles são bem mais velhos que você, assim você pode adivinhar quem vai eventualmente 
	``expirar''  primeiro (morrer).
	A má notícia é que é que eles são mais experientes nesse jogo (afinal, eles perseveraram na frente da banca deles, não!?).
	
\end{itemize}
	
Aqui estão algumas linhas-guia que podem ajudá-lo quando você finalmente levar a sério escrever. 
A lista segue infinitamente; você provavelmente não vai querer ler isso tudo de uma vez. Mas, por favor, leia isso antes de 
escrever qualquer coisa.

\hfill

\section{A Idéia Geral}	

\begin{enumerate}
	
	\item Uma monografia é um documento formal onde o aluno descreve a realização de um trabalho técnico onde usou as técnicas e conceitos aprendidos durante o curso de 	graduação.

	\item Em geral, toda afirmação em uma monografia deve ser embasada ou por uma referência em literatura científica publicada ou por um trabalho original.
	Acima de tudo, uma monografia não repete os detalhes do pensamento crítico e análises encontradas nas fontes publicadas; usa o resultado como fato e 
	referencia o leitor às fontes para mais detalhes. 

	\item Cada sentença em uma monografia deve ser completa e correta gramaticalmente. Além do mais, a monografia deve satisfazer estritamente as 
	regras da gramática formal (e.x., sem contrações, sem coloquialismo, sem pronúncias erradas, sem jargão técnico indefinido, sem piadas escondidas e sem gírias,
	mesmo quando tais termos ou frases são de comum uso na língua falada). Realmente, a escrita de uma monografia deve ser um cristal limpo.
	Sombras de significados importam; a terminologia e a prosa devem fazer uma fina distinção. As palavras devem carregar exatamente o sentido pretendido, nada mais e nada 	menos.

	\item Cada afirmação em uma monografia deve ser correta e defensível no sentido lógico e científico. Acima de tudo, as discussões em uma monografia devem satisfazer
	a maioria das estritas regras de lógica aplicada à matemática e Engenharia.                       

\end{enumerate}

\section{O que se Deve Aprender do Exercício}

\begin{enumerate}

	\item Todo engenheiro precisa comunicar descobertas; a monografia fornece um treinamento para comunicação com outros engenheiros.

	\item Escrever uma monografia requer que o estudante pense profundamente, para organizar a discussão técnica, para reunir argumentos que convencerão outros engenheiro,
	e seguir as regras para uma rigorosa apresentação dos argumentos e discussões.

\end{enumerate}

\section{Regra do Polegar}

Boa escrita é essencial para uma monografia. Entretanto, boa escrita não pode compensar uma escassez de idéias ou conceitos.
Pelo contrário, uma apresentação limpa sempre expõe fraquezas.

\section{Definições e Terminologia}

\begin{enumerate}

	\item Cada termo técnico usado em uma monografia deve ser definido ou por uma referência à uma definição publicada anteriormente (para termos padrões com seus significados usuais)
	ou por uma precisa, não-ambígua definição que aparece antes do termo ser usado (para termos novos ou um termo padrão usado de maneira não usual).

	\item Cada termo deve ser usado de uma e única maneira por toda monografia.			

	\item A forma mais fácil de evitar uma longa série de definições é incluir uma afirmação: ``a terminologia usada no decorrer deste documento segue a mesma dada em [CITAÇÃO].''
	Então, só defina exceções.

	\item O capítulo introdutório pode dar o intuito (i.e., definições informais) dos termos fornecidos, os quais serão mais precisamente definidos depois.

\end{enumerate}

\section{Termos e Frase a Evitar}
	
\begin{itemize}

	\item Advérbios

		\indent Na maioria das vezes, são ``muito frequentemente usados demais''. Ao invés deles use palavras mais fortes. Alguém pode dizer, por exemplo, `` Escritores abusam de advérbios.''
	
	\item Piadas ou Trocadilhos

		\indent Esses não têm lugar em um documento formal.
	
	\item ``Ruim'', ``Bom'', ``Terrível'', ``Estúpido''

		\indent Uma monografia não faz julgamento moral. Use ``incorreto/correto'' para se referir à erros ou corretudes de fato.
		Use palavras precisas ou frases para avaliar qualidade (e.x. ``método A requer menos recurso computacional que método B''). Em geral, deve-se evitar todos os julgamentos qualitativos.
		
	\item ``Verdade'', ``Puro''

		\indent No mesmo senso de ``bom'' (é um julgamento).
		
	\item ``Perfeito''

		\indent Nada é.
		
	\item ``Uma solução ideal''

		\indent Você está julgando de novo.
		
	\item ``Hoje'', ``Tempos modernos''

		\indent Hoje é o ontem de amanhã.
		
	\item ``Logo''

		\indent Logo quanto? Hoje à noite? Próxima década?
		
	\item ``Estávamos surpresos ao ver ...''

		\indent Mesmo se você estivesse, e daí?
		
	\item ``Parece'', ``Aparentemente''

		\indent Não importa como algo aparenta.
	
	\item ``Parece mostrar''

		\indent Tudo o que importa são os fatos.
		
	\item ``Em termos de''

		\indent Normalmente vago.
		
	\item ``Baseado em'', ``X-baseado'', ``Como base de''

		\indent Cuidado, pode ser vago.
		
	\item ``Diferente''

		\indent	Não significa ``vários''. Diferente do que?
		
	\item ``Na luz de''

		\indent	Coloquial. 
		
	\item ``Um monte de''

		\indent	Vago & Coloquial
		
	\item ``Tipo de''

		\indent	Vago & Coloquial
		
	\item ``Algo como''

		\indent	Vago & Coloquial
		
	\item ``Mais ou Menos'' 

		\indent	Vago & Coloquial
		
	\item ``Número de''

		\indent	Vago, você quer dizer, ``alguns'', ``muitos'' ou ``a maioria''? Uma afirmação quantitativa é preferível.
		
	\item ``Devido a''

		\indent	Coloquial
		
	\item ``Provavelmente''

		\indent	Apenas se você souber a probabilidade estatística (se você sabe, afirme quantitativamente).
		
	\item ``Obviamente'', ``Claramente''

		\indent Tenha cuidado: Óbvio/Claro para todos?
		
	\item ``Simples''

		\indent Pode ter uma conotação negativa, como em ``simplório''.
		
	\item ``Junto com''

		\indent Use somente ``com''.
		
	\item ``Na verdade'', ``Realmente''

		\indent Defina os termos claramente para eliminar a necessidade de esclarecimento.
		
	\item ``O fato de''

		\indent Faz uma meta-sentença; reformule a frase.
		
	\item ``Isso'', ``Aquilo''

		\indent Como em ``Estas causa envolvem.'' Razão: ``Isso'' pode referir ao sujeito da sentença anterior, à toda sentença anterior, todo o parágrafo anterior, toda a seção anterior, etc. 
		Mais importante, pode ser interpretado no sentido correto ou no meta-sentido. Por exemplo: \textit{``X faz Y. Isso significa...''} o leitor pode assumir ``isso'' referindo ao 			
		\textit	{Y} ou ao fato de \textit{X faz Y}. Mesmo quando restrito (e.x., ``esta computação''), a frase é fraca e frequentemente ambígua.
		
	\item ``Você irá ler sobre isso''

		\indent A segunda pessoa não tem lugar em uma monografia.
		
	\item ``Eu vou descrever''

		\indent A primeira pessoa não tem lugar em uma monografia formal. Se auto-referência é essencial, escreva como ``Seção 10 descreve...''
		
	\item ``Nós'' como em ``Vemos que''

		\indent Uma armadilha a evitar. Razão: Quase toda sentença pode ser escrita para começar com ``nós'' porque `nós'' pode se referir: ao leitor e autor, ao autor e consultor, ao autor e grupo de pesquisa, engenheiros de computação, a toda comunidade de Engenharia da computação, ou algum outro grupo não especificado. 

	\item ``Esperançosamente, o programa''

		\indent Programas não tem esperança, não até serem implementados com sistemas de IA. Aliás, se você estiver escrevendo uma tese de IA, fale com outra pessoa: pessoas de IA tem seus próprios sistemas de regras.
		
	\item ``...um famoso pesquisador...''

		\indent Não importa quem disse ou fez. De fato, tais afirmações prejudicam o leitor.
		
	\item Tenha cuidado usando: ``poucos, maioria, todos, algum, cada'' 

		\indent Uma monografia é precisa. Se a sentença diz ``Maioria dos sistemas computacionais contém X'', você deve ser capaz de defender isso. Você tem certeza que conhece os fatos? Quantos computadores foram construídos e vendidos ontem?
		
	\item ``Deve'', ``Sempre''

		\indent Absolutamente?
		
	\item ``Deveria''

		\indent Quem disse isso?
		
	\item ``Prova'', ``Comprova''

		\indent Um matemático aceitaria que isso é uma prova?

	\item ``Pode'', ``Poderia''

		\indent Sua mãe provavelmente lhe disse a diferença.

\end{itemize}

\section{Voz}

Use construções ativas. Por exemplo, diga ``o sistema operacional inicia o dispositivo'' ao invés de ``o dispositivo é iniciado pelo sistema operacional.''

\section{Tempo Verbal}

Escreva no presente. ``O sistema escreve a página no disco e então usa o frame...'' ao invés de ``O sistema usará o frame depois de ter escrito a página no disco...''

\section{Defina Negações com Antecedência}

Exemplo: diga ``Nenhum bloco de dados espera na fila de saída'' ao invés de ``Um bloco de dados esperando saída não está na fila.''

\section{Gramática e Lógica}

Tenha cuidado pois o sujeito de cada sentença realmente faz o que o verbo diz q ele faz. 
Dizer ``Programas devem fazer chamada de processo usando a instrução X'' não é o mesmo que dizer ``Programas devem usar a instrução X quando chamam um procedimento.'' 
De fato, a primeira é evidentemente falsa! 
Outro exemplo: ``RPC requer programas para transmitir pacotes grandes'' não é o mesmo que ``RPC requer um mecanismo que permita programas transmitirem pacotes grandes.''

\section{Foco nos Resultados e não nas Pessoas/Circunstâncias em que Foram Obtidos}
		
``Depois de trabalhar oito horas no laboratório naquela noite. nós percebemos...'' não tem lugar na monografia. 
Não importa quando você percebeu isso, ou quanto tempo você trabalhou para obter a resposta. 
Outro exemplo: ``Jim e eu chegamos aos números mostrados na tabela 3 medindo...'' Ponha um agradecimento para Jim na monografia, mas não inclua nomes (nem mesmo o seu) no corpo principal. 
Você pode estar tentado a documentar uma longa série de experimentos que não produziram nada ou uma coincidência que resultou em sucesso. Evite completamente isso. 
Em particular, não documente aparentemente influências místicas (e.x., ``se aquele gato não tivesse rastejado pelo buraco no chão, poderíamos 
não ter descoberto o indicador de erro do fornecimento de energia na ponte de rede''). Nunca atribua tais eventos à causas místicas ou dê a 
entender que forças estranhas podem ter afetado seu resultado. 
Resumo: Prenda-se nos fatos evidentes. Descreva os resultados sem mencionar suas reações ou eventos que o ajudaram a alcançá-los.

\section{Evite Auto-Avaliação (Elogio e Crítica)}

Ambos os exemplos a seguir estão incorretos: ``O método esboçado na Seção 2 representa o maior avanço em design de sistemas distribuídos porque...'' 
``Embora a técnica na próxima seção não seja extraordinário,...''

\section{Referências à Trabalhos}

Sempre cita-se o artigo, não o autor. Assim, usa-se um verbo no singular para referir ao artigo, mesmo que tenha muitos autores. Por exemplo ``Johnson e Smith [Johnson and Smith1995] relata que...''
Evite a frase `` os autores afirmam que X''. O uso de ``afirmam'' lança dúvida em ``X'' porque referencia os pensamentos do autor ao invés dos fatos. Se você concorda ``X'' está correto, simplesmente escreva ``X'' seguido da referência. Se absolutamente deve referenciar um artigo ao invés do resultado, diga ``o artigo afirma que'' ou ``Johnson e Smith [Johnson and Smith1995] apresentam evidências que...''

\section{Conceito Vs. Exemplo}

Um leitor pode ficar confuso quando um conceito e um exemplo deste estão embaçados. 
Exemplos comuns incluem: um algoritmo e um programa particular que o implementa, uma linguagem de programação e um compilador, uma abstração geral e sua implementação 
particular em um sistema de computador, uma estrutura de dados e uma instância particular em memória.

\section{Terminologia para Conceitos e Abstrações}

Quando definir a terminologia para um conceito, tenha cuidado para decidir precisamente como a idéia se traduz para uma implementação. Considere a seguinte discussão:
\textit{Sistemas VM incluem um conceito conhecido como endereço de espaço. O sistema cria dinamicamente um endereço de espaço quando um programa precisa de um, e destrói um endereço de espaço 
quando o programa que criou o espaço terminar de usá-lo. Um sistema VM usa um pequeno, finito número para identificar cada endereço de espaço. 
Conceitualmente, entende-se que cada endereço de espaço deveria ter um novo identificador. Entretanto, se um sistema VM executa por um tempo que esgote todos os possíveis identificadores de endereços de espaço, 
ele deve reusar um número}
O ponto importante é que a discussão só faz sentido porque define ``endereço de espaço'' independente de ``identificador de endereço de espaço''. 
Se espera-se discutir as diferenças entre um conceito e sua implementação, as definições devem permitir tal distinção.

\section{Conhecimento Vs. Dados}

O fato que resulta de um experimento é chamado ``dado''. 
O termo ``conhecimento'' implica que o fato tenha sido analisado, condensado ou combinado com fatos de outros experimentos para produzir informação útil.

\section{Causa e Efeito}

Uma monografia deve separar cuidadosamente causa-efeito de simples correlações estatísticas. Por exemplo, mesmo se todos os programas de computador escritos no laboratório do Professor X requerem mais memória que os programas escritos no laboratório do Professor Y, isso pode não ter nada a ver com os professores ou laboratórios ou programadores (e.x., talvez as pessoas que trabalham no laboratório do Professor X estejam trabalhando em aplicações que requerem mais memória do que no laboratório do Professor Y).

\section{Descreva Somente Conclusões Comprovadas}   		

Deve-se ter cuidado para apenas escrever conclusões que as evidências suportam. Por exemplo, se programas executam muito mais lento no computador A do que no computador B, 
não pode-se concluir que o processador de A é mais lento que o de B a menos que se tenha anotado todas as diferenças entre os sistemas operacionais dos computadores, dispositivos de entrada e saída, 
tamanho de memória, memória cache, ou largura de banda do barramento interno. 
De fato, deve-se ainda abster-se de julgamentos a menos que se tenha os resultados de um experimento controlado (e.x., executando uma lista de vários programas muitas vezes, cada um quando o computador 
estiver ocioso). Mesmo se a causa de algum fenômeno parece óbvia, não pode-se dar uma conclusão sem sólida evidência embasada.

\section{Comércio e Ciência}

Em uma monografia, nunca se escreve conclusões sobre viabilidade econômica ou sucesso comercial de uma idéia/método, nem faz-se especulações sobre a história do desenvolvimento ou origens de
uma idéia. Um engenheiro deve permanecer objetivo sobre os méritos de uma idéia, independente de sua popularidade comercial. Em particular, um engenheiro nunca assume que o sucesso comercial é uma medida válida 
de mérito (muitos produtos populares não são nem bem projetados nem bem construídos). Assim, afirmações tais como ``mais de quatrocentos vendedores fazem produtos usando a técnica Y'' são irrelevantes em uma monografia.

\section{Política e Ciência}

Um engenheiro evita toda influência política quando está avaliando Idéias. Obviamente, não deveria importar se grupos governamentais, grupos políticos, grupos religiosos ou outras organizações aprovam uma idéia.
Mais importante e frequentemente despercebida, não importa se uma idéia uma idéia foi originada por um engenheiro que já tenha ganho um premio Nobel ou um aluno no primeiro ano de graduação. Deve-se avaliar a idéia independente da fonte.

\section{Organizações Canônicas}
 	
Em geral toda monografia deve definir o problema que motivou a pesquisa, contar por que este problema é importante, contar o que outros fizeram, descrever as novas contribuições, documentar os experimentos que validam a contribuição e fazer conclusões.
Não existe organização canônica para uma monografia; cada uma é única. Entretanto, novatos que escrevem uma monografia em uma área experimental da Engenharia da Computação podem achar os seguintes exemplos um bom ponto de início: 

\begin{itemize}
	
	\item \textbf{Capítulo 1: Introdução}

	Uma visão do problema; por que isso é importante; um resumo de um trabalho já existente e uma afirmação de suas hipóteses ou questões específicas a serem exploradas. Faça com que seja legível pra qualquer um.

	\item \textbf{Capítulo 2: Definições}

	Somente termos novos. Faça as definições precisas, concisas e não-ambíguas.

	\item \textbf{Capítulo 3: Modelo Conceitual}

	Descreva o conceito central que influencia o seu trabalho. Faça disso um ``tema'' que amarram todos os seus argumentos. Isso deveria fornecer uma resposta para a questão apresentada na introdução em um nível conceitual. Se necessário, adicione outro capítulo para dar um raciocínio adicional sobre o problema ou sua solução.

	\item \textbf{Capítulo 4: Medidas Experimentais}

	Descreva o resultado experimental que forneça evidências para embasar sua tese. Normalmente experimentos enfatizam prova-ou-conceito (demonstrando a viabilidade de um método/técnica) ou eficiência (demonstrando que um método/técnica proporciona uma performance melhor do que as que já existem)
	
	\item \textbf{Capítulo 5: Resultados e Consequências}

	Descreva variações, extensões ou outras aplicações da idéia central.

	\item \textbf{Capítulo 6: Conclusões}

	Resumo do que foi aprendido e como isso pode ser aplicado. Mencione as possibilidades para pesquisas futuras.

	\item \textbf{Resumo/Abstract}

	Um pequeno (poucos parágrafos) resumo da monografia. Descreva o problema e a abordagem da pesquisa. Enfatize as contribuições originais.

\end{itemize}

\section{Ordem Sugerida para Escrever}

A maneira mais fácil de construir uma monografia é de dentro para fora. Comece escrevendo os capítulos que descrevem sua pesquisa (3, 4 e 5 nas linhas acima). Colete termos como eles surgem no texto e dê uma definição para cada um.
Defina cada termo técnico,  mesmo que você o use de maneira convencional.

Organize as definições em um capítulo separado. Faça as definições precisas e formais. Reveja depois os capítulos para verificar que cada uso de termo técnico adere à uma definição. Depois de ler os capítulos do meio para verificar terminologia, escreva a conclusão. escreva a Introdução logo depois da Conclusão. Finalmente, complete o resumo/abstract.

\section{A Chave do Sucesso}

Aliás, existe uma chave para o sucesso: prática. ninguém nunca aprendeu a escrever lendo composições como esta. Ao invés disso, você precisa praticar, praticar, praticar. Todo dia.

\section{Pensamentos de Despedida}

Nos despedimos de você com as seguintes Idéias para meditar. Se não significarem nada para você agora, visite-as novamente depois de ter escrito sua monografia.

	

	\indent\indent Depois de grande dor, chega um pensamento formal.

		\indent \indent \indent --Emily Dickinson

		
	\indent\indent Um homem pode escrever a qualquer hora, se ele se mantiver persistente para tal.

		\indent \indent \indent --Samuel Johnson

		
	\indent\indent Permaneça perfeito até o final da estrada.

		\indent \indent \indent --Harry Lauder

	 
	\indent\indent Uma típica tese de Ph.D. é nada mais que transferir ossos de um cemitério para outro.

		\indent \indent \indent --Frank J. Dobie

	
	
	
	
	
	
	
	








	
	
	
	
	
	
	
	
	
	
	
	
	
	
